\documentclass{scrbook}

%%~~~~~~~~~~~~~~~~~~~~~~~~~~~~~~~~~~~~~~~~~~~~~~~~~~~~~~~~~~~~
%% Layout

\KOMAoptions{
  paper=A4,
  BCOR=0mm,
  pagesize,
  mpinclude=true,               % Treat margin as part of text body
  fontsize=11pt,
  captions=nooneline,           % Don’t center one-line captions
}

%% Recompute layout based on the options above.  Without it, the layout is
%% pretty bad and there are a bunch of warnings in the output.
\recalctypearea

%% More space for margin notes.
\setlength{\marginparwidth}{3\marginparwidth}

%% More space for main text body.
\addtolength{\textheight}{5em}

%% Small space around and below floats.
\setlength{\intextsep}{1ex}

%%~~~~~~~~~~~~~~~~~~~~~~~~~~~~~~~~~~~~~~~~~~~~~~~~~~~~~~~~~~~~
%% General typography

%% TTF fonts with XeLaTeX.
\usepackage{fontspec}

%% Charter is an amazing body font.
\setromanfont{Charter}

%% Fira Sans is nice for the titles.
\setsansfont[Scale=MatchUppercase]{Fira Sans}

%% A monospace font that handles utf8 chars.  And is not too wide.
%% Scaled to match the body font.
\setmonofont[Scale=MatchUppercase]{Source Code Pro}

%% French typographical conventions and translated names for “Chapitre”, “Table
%% des matières”, etc.
\usepackage[french]{babel}

%% Move characters around and minimize hyphenation.  Good for diversity.  Not as
%% powerful with XeLaTeX, as font information may be missing.
\usepackage{microtype}

%%~~~~~~~~~~~~~~~~~~~~~~~~~~~~~~~~~~~~~~~~~~~~~~~~~~~~~~~~~~~~
%% Side notes (in the outer margin)

%% Font to use for side notes.
\newkomafont{sidenote}{\footnotesize}

%% For hyphenation in ragged text, which gives more even side notes.
\usepackage{ragged2e}

%% Custom commands for side notes with a counter and ragged text side
%% following odd and even pages.
\newcounter{sidenotecounter}
\newcommand{\sidenotemark}{\textsuperscript{\thesidenotecounter}}
\newcommand{\sidenote}[1]{%
  \marginpar[\RaggedLeft\usekomafont{sidenote}{\sidenotemark#1\par}]%
            {\RaggedRight\usekomafont{sidenote}{\sidenotemark#1\par}}}

%% Replace all footnotes with a marginnote.
\renewcommand{\footnote}{\refstepcounter{sidenotecounter}\sidenotemark\sidenote}

%% I want to put caption of figures in the margin as well.  Tufte-latex does
%% that, but also does plenty of stuff I don’t need (like redefining my layout
%% with ‘geometry’).  I load a custom version of the tufte-latex package with
%% most of the useless stuff commented out.
%% \usepackage[
%%   nobib,                        % Don’t override \cite
%%   symmetric,                    % Twosides
%%   sidenote=raggedouter,         % Always ragged text for margin stuff
%%   marginnote=raggedouter,
%%   caption=raggedouter,
%%   citation=raggedouter,
%%   marginals=raggedouter,
%% ]{tufte-latex}

%% KOMAscript indents captions.  I don’t want that for margin captions.
\setcapindent{0pt}
%% \setkomafont{caption}{\FloatRaggedOuter\footnotesize}

%% Babel french puts it in small caps.  I don’t want that.
%% Additionally, override figurename for listings.
%% HACK: maybe using a new environment for the listings would be a better
%% solution.
\newcommand{\UnsetListingFigureName}{\renewcaptionname{french}{\figurename}{Figure}}
\UnsetListingFigureName
\newcommand{\SetListingFigureName}{\renewcaptionname{french}{\figurename}{Code}}

%%~~~~~~~~~~~~~~~~~~~~~~~~~~~~~~~~~~~~~~~~~~~~~~~~~~~~~~~~~~~~
%% Source code listings

%% Environment for source code snippets.
\usepackage{listings}
\lstset{
  basicstyle=\ttfamily\small,
  commentstyle=\ttfamily,
  columns=fullflexible, keepspaces=true,
  breaklines=false, showstringspaces=false,
  escapeinside={//*}{\^^M},     % Escape to LaTeX between //* and line return
  aboveskip=-4pt,               % No space above and below, symmetric
  belowskip=0pt,
  %extendedchars=true, inputencoding=utf8,
}

%% Org export produces environment for ‘js’, so we must define that language for
%% listings.
\lstdefinelanguage{js}{
  language={Java},
  morekeywords={with,var,function},
  deletekeywords={double},
}

\lstdefinelanguage{smalltalk}{}

%% Accept ‘diff’ as language, no special treatment.
\lstdefinelanguage{diff}{}

%%~~~~~~~~~~~~~~~~~~~~~~~~~~~~~~~~~~~~~~~~~~~~~~~~~~~~~~~~~~~~
%% Bibliography

%% Quells a warning from biblatex with babel activated.  Not sure /what/ it does
%% though.
\usepackage{csquotes}

%% Generates the bibliography.  Handles UTF8-encoded bib files.
\usepackage[
  backend=biber,
  firstinits=false,
  sorting=nyt,                  % Sort by name, year, title
  backref=true,
  style=alphabetic,
  maxbibnames=10
]{biblatex}
\addbibresource{refs.bib}

%%~~~~~~~~~~~~~~~~~~~~~~~~~~~~~~~~~~~~~~~~~~~~~~~~~~~~~~~~~~~~
%% Others

%% Needed for the \text command in math-mode.
\usepackage{amsmath}

%% A touch of color
\usepackage{xcolor}
\definecolor{rubric}{rgb}{0.65,0.12,0.09}
\definecolor{azure}{rgb}{0.06,0.3,0.5}

%% Hyperlinks for citations and external links.
\usepackage{hyperref}
\hypersetup{
  unicode,                      % Always a good idea?
  hyperfootnotes=false,         % Footnotes are broken
  xetex,                        % Might as well tell it already
  colorlinks=true,              % Color links rather than put boxes around them
  linkcolor=rubric,             % internal link
  citecolor=rubric,             % bibliography link
  urlcolor=azure,               % external link
}


%% Temporary handling of special blocks.
\newenvironment{epigraph}{}{}
\newenvironment{aside}{}{}
\newenvironment{side-figure}{}{}
\newenvironment{full-figure}{}{}
