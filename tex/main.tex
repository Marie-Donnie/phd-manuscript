\documentclass{scrbook}

%%~~~~~~~~~~~~~~~~~~~~~~~~~~~~~~~~~~~~~~~~~~~~~~~~~~~~~~~~~~~~
%% Layout

\KOMAoptions{
  paper=A4,
  BCOR=0mm,
  pagesize,
  mpinclude=true,               % Treat margin as part of text body
  fontsize=11pt,
}

%% Recompute layout based on the options above.  Without it, the layout is
%% pretty bad and there are a bunch of warnings in the output.
\recalctypearea

%% More space for margin notes.
\setlength{\marginparwidth}{3\marginparwidth}

%% More space for main text body.
\addtolength{\textheight}{5em}

%%~~~~~~~~~~~~~~~~~~~~~~~~~~~~~~~~~~~~~~~~~~~~~~~~~~~~~~~~~~~~
%% General typography

%% TTF fonts with XeLaTeX.
\usepackage{fontspec}

%% A monospace font that handles utf8 chars.  And is not too wide.
\setmonofont{Ubuntu Mono}

%% French typographical conventions and translated names for “Chapitre”, “Table
%% des matières”, etc.
\usepackage[french]{babel}

%% Move characters around and minimize hyphenation.  Good for diversity.  Not as
%% powerful with XeLaTeX, as font information may be missing.
\usepackage{microtype}

%%~~~~~~~~~~~~~~~~~~~~~~~~~~~~~~~~~~~~~~~~~~~~~~~~~~~~~~~~~~~~
%% Margin notes

%% Font to use for margin notes.
\newkomafont{margin}{\footnotesize}

%% For hyphenation in ragged text, which gives more even margin notes.
\usepackage{ragged2e}

%% Custom commands for margin notes with a counter and ragged text side
%% following odd and even pages.
\newcounter{marginnotecounter}
\newcommand{\marginnotemark}{\textsuperscript{\themarginnotecounter}}
\newcommand{\marginnote}[1]{%
  \marginpar[\RaggedLeft\usekomafont{margin}{\marginnotemark#1\par}]%
            {\RaggedRight\usekomafont{margin}{\marginnotemark#1\par}}}

%% Replace all footnotes with a marginnote.
\renewcommand{\footnote}{\refstepcounter{marginnotecounter}\marginnotemark\marginnote}

%% I want to put caption of figures in the margin as well.  Tufte-latex does
%% that, but also does plenty of stuff I don’t need (like redefining my layout
%% with ‘geometry’).  I load a custom version of the tufte-latex package with
%% most of the useless stuff commented out.
\usepackage[
  nobib,                        % Don’t override \cite
  symmetric,                    % Twosides
  sidenote=raggedouter  ,       % Always ragged text for margin stuff
  marginnote=raggedouter,
  caption=raggedouter,
  citation=raggedouter,
  marginals=raggedouter,
]{tufte-latex}

%%~~~~~~~~~~~~~~~~~~~~~~~~~~~~~~~~~~~~~~~~~~~~~~~~~~~~~~~~~~~~
%% Source code listings

%% Environment for source code snippets.
\usepackage{listings}
\lstset{
  basicstyle=\ttfamily\small,
  commentstyle=\ttfamily,
  columns=fullflexible, keepspaces=true,
  breaklines=false, showstringspaces=false,
  escapeinside={//*}{\^^M},     % Escape to LaTeX between //* and line return
  captionpos=b,
  %extendedchars=true, inputencoding=utf8,
}

%% Org export produces environment for ‘js’, so we must define that language for
%% listings.
\lstdefinelanguage{js}{
  language={Java},
  morekeywords={with,var,function},
  deletekeywords={double},
}

%% Accept ‘diff’ as language, no special treatment.
\lstdefinelanguage{diff}{}

%%~~~~~~~~~~~~~~~~~~~~~~~~~~~~~~~~~~~~~~~~~~~~~~~~~~~~~~~~~~~~
%% Bibliography

%% Quells a warning from biblatex with babel activated.  Not sure /what/ it does
%% though.
\usepackage{csquotes}

%% Generates the bibliography.  Handles UTF8-encoded bib files.
\usepackage[
  backend=biber,
  firstinits=false,
  sorting=nyt,                  % Sort by name, year, title
  backref=true,
  style=alphabetic,
  maxbibnames=10
]{biblatex}
\addbibresource{../refs.bib}

%%~~~~~~~~~~~~~~~~~~~~~~~~~~~~~~~~~~~~~~~~~~~~~~~~~~~~~~~~~~~~
%% Others

%% A touch of color
\usepackage{xcolor}
\definecolor{rubric}{rgb}{0.65,0.12,0.09}
\definecolor{azure}{rgb}{0.06,0.3,0.5}

%% Hyperlinks for citations and external links.
\usepackage{hyperref}
\hypersetup{
  unicode,                      % Always a good idea?
  hyperfootnotes=false,         % Footnotes are broken
  xetex,                        % Might as well tell it already
  colorlinks=true,              % Color links rather than put boxes around them
  linkcolor=rubric,             % internal link
  citecolor=rubric,             % bibliography link
  urlcolor=azure,               % external link
}


\newcommand{\titlefr}{Étendre des interpréteurs avec des mécanismes de langage}
\newcommand{\subtitlefr}{Comment j'ai appris à ne plus m'en faire et à aimer le code}
\newcommand{\titleen}{Extending interpreters with language mechanisms}

\title{\titlefr}
\author{fmdkdd}

\begin{document}

%% Insert title, TOC, and acknowledgments.  The Org document must not insert
%% title or TOC itself, otherwise they will appear before the frontmatter
%% command.

\frontmatter

\begin{titlingpage}
\raggedright
\newgeometry{left=2cm,right=6cm,top=3cm,bottom=2cm}
\sffamily

{\noindent\Huge\titlefr}

\vspace{1.5cm}

\LARGE
\noindent
Florent \textsc{Marchand de Kerchove}

\vfill

\begin{raggedright}
\rmfamily\normalsize
\itshape
Thèse de doctorat présentée en vue de l'obtention du
grade de docteur de l'université de Nantes
sous le label de l'université de Nantes, Angers, Le Mans.

\vspace{1em}
École doctorale: sciences et technologies de l'information et mathématiques\\
Discipline: informatique, section CNU 27\\
Unité de recherche: laboratoire d'informatique de Nantes-Atlantique (LINA)\\

\vspace{1em}
Soutenue le 27 octobre 2016, devant le jury composé de:

\vspace{0.5em}
M. Hulu Berlu, professeur, Tartifrice, rapporteur;\\
M. Ali Barli, professeur, Hamilcar, rapporteur;\\
M. Po Nan, professeur, Asifond, examinateur;\\
M. Jacques Noyé, maître-assistant, Mines Nantes, encadrant de thèse;\\
M. Mario Südholt, professeur, Mines Nantes, directeur de thèse.\\
\end{raggedright}
\restoregeometry
\end{titlingpage}

\newgeometry{left=3cm,right=4.4cm}
\pagestyle{empty}
\tableofcontents*
\clearpage
\restoregeometry

%% \chapter*{Remerciements}
%% %% Fix heading mark showing “Contents”
%% %% as suggested here:
%% %% https://en.wikibooks.org/wiki/LaTeX/Document_Structure#The_document_environment
%% \markboth{\MakeUppercase{Remerciements}}{}
%% \input{acks}

\pagestyle{headings}
\mainmatter
\input{manuscript}

%% Back matter for bibliography, eventual index.
\backmatter

\newgeometry{left=2cm,right=3cm}
\printbibliography
\restoregeometry

%% Back cover
\newcommand{\heading}[1]{{\sffamily\textbf{#1}\vspace{4pt}}}

\newgeometry{left=1cm,right=1cm,top=2cm}
\begin{titlingpage}
\vspace{2cm}
%% \begin{addmargin}[-1cm]{-4cm}
\begin{flushright}
\sffamily\LARGE\titleen
\end{flushright}
\vfill
\noindent
\begin{minipage}[t]{9cm}
\heading{Résumé}\\
\input{abstract-fr}
\end{minipage}%
\hspace{1cm}%
\begin{minipage}[t]{9cm}
\heading{Abstract}\\
\input{abstract-en}
\end{minipage}

\vfill
\noindent
\begin{minipage}[t]{9cm}
\heading{Mots-clés}\\
Extensibilité, modularité, interpréteurs, instrumentation, patrons de
conception, programmation par objets, programmation par aspects, JavaScript,
langages de programmation.
\end{minipage}%
\hspace{1cm}
\begin{minipage}[t]{9cm}
\heading{Keywords}\\
Extensibility, modularity, interpreters, instrumentation, Design Patterns,
Object-Oriented Programming, Aspect-Oriented Programming, JavaScript,
programming languages.
\end{minipage}
%% \end{addmargin}
\end{titlingpage}

\end{document}
